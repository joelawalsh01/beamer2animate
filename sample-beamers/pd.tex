\documentclass[aspectratio=169]{beamer}
\usetheme{Madrid}
\usecolortheme{default}

\usepackage{amsmath, amssymb}
\usepackage{graphicx}
\usepackage{listings}
\usepackage{xcolor}
\usepackage{upquote}

% Code listing style
\lstset{
    language=Python,
    basicstyle=\ttfamily\small,
    keywordstyle=\color{blue},
    stringstyle=\color{red},
    commentstyle=\color{gray},
    frame=single,
    breaklines=true,
    showstringspaces=false
}

\title{Partial Derivatives: A Visual Review}
\subtitle{COMP 395 -- Deep Learning}
\author{}
\date{}

\begin{document}

%-----------------------------------------------
\begin{frame}
\titlepage
\vfill
\begin{center}
\footnotesize\texttt{pip install numpy matplotlib}
\end{center}
\end{frame}

%-----------------------------------------------
\begin{frame}{From One Variable to Many}

\textbf{Single-variable review:} Given $f(x) = x^3$, the derivative $\frac{df}{dx} = 3x^2$ tells us the rate of change of $f$ as $x$ changes.

\vspace{0.5cm}

\textbf{But what if we have more than one variable?}

\vspace{0.3cm}

Consider $f(x, y) = x^2 + y^2$

\begin{itemize}
    \item This defines a \textbf{surface} in 3D space
    \item At any point, $f$ can change in the $x$-direction, the $y$-direction, or both
    \item We need a way to measure each rate of change \textit{separately}
\end{itemize}

\vspace{0.3cm}

\textbf{Key idea:} Hold all variables constant except one, then differentiate.

\end{frame}

%-----------------------------------------------
\begin{frame}{Partial Derivative Definition}

\textbf{Notation:} For $f(x, y)$, we write:
\begin{itemize}
    \item $\frac{\partial f}{\partial x}$ -- the partial derivative with respect to $x$
    \item $\frac{\partial f}{\partial y}$ -- the partial derivative with respect to $y$
\end{itemize}

\vspace{0.5cm}

\textbf{How to compute $\frac{\partial f}{\partial x}$:}
\begin{enumerate}
    \item Treat $y$ as a \textbf{constant}
    \item Differentiate with respect to $x$ using your usual rules
\end{enumerate}

\vspace{0.5cm}

\textbf{How to compute $\frac{\partial f}{\partial y}$:}
\begin{enumerate}
    \item Treat $x$ as a \textbf{constant}
    \item Differentiate with respect to $y$ using your usual rules
\end{enumerate}

\end{frame}

%-----------------------------------------------
\begin{frame}{Example 1: Power Rule}

Let $f(x, y) = x^2 + 3xy + y^3$

\vspace{0.5cm}

\textbf{Find $\frac{\partial f}{\partial x}$:} (treat $y$ as constant)
\begin{align*}
\frac{\partial f}{\partial x} &= \frac{\partial}{\partial x}(x^2) + \frac{\partial}{\partial x}(3xy) + \frac{\partial}{\partial x}(y^3) \\
&= 2x + 3y + 0 \\
&= 2x + 3y
\end{align*}

\vspace{0.3cm}

\textbf{Find $\frac{\partial f}{\partial y}$:} (treat $x$ as constant)
\begin{align*}
\frac{\partial f}{\partial y} &= \frac{\partial}{\partial y}(x^2) + \frac{\partial}{\partial y}(3xy) + \frac{\partial}{\partial y}(y^3) \\
&= 0 + 3x + 3y^2 \\
&= 3x + 3y^2
\end{align*}

\end{frame}

%-----------------------------------------------
\begin{frame}{Example 2: With Logarithms and Trig}

Let $g(x, y) = x^2 \ln(y) + \sin(x)$

\vspace{0.5cm}

\textbf{Find $\frac{\partial g}{\partial x}$:} (treat $y$ as constant, so $\ln(y)$ is just a constant)
\begin{align*}
\frac{\partial g}{\partial x} &= 2x \cdot \ln(y) + \cos(x)
\end{align*}

\vspace{0.5cm}

\textbf{Find $\frac{\partial g}{\partial y}$:} (treat $x$ as constant, so $x^2$ and $\sin(x)$ are constants)
\begin{align*}
\frac{\partial g}{\partial y} &= x^2 \cdot \frac{1}{y} + 0 = \frac{x^2}{y}
\end{align*}

\end{frame}

%-----------------------------------------------
\begin{frame}{Visualizing the Surface}

Consider $f(x, y) = x^2 + y^2$

\vspace{0.3cm}

This is a \textbf{paraboloid} -- a bowl-shaped surface in 3D.

\vspace{0.3cm}

\begin{columns}
\column{0.5\textwidth}
\textbf{What does $\frac{\partial f}{\partial x}$ mean geometrically?}

\vspace{0.2cm}

\begin{itemize}
    \item Slice the surface with a plane where $y = c$ (constant)
    \item You get a 2D curve (a parabola)
    \item $\frac{\partial f}{\partial x}$ is the slope of that curve
\end{itemize}

\column{0.5\textwidth}
\textbf{What does $\frac{\partial f}{\partial y}$ mean geometrically?}

\vspace{0.2cm}

\begin{itemize}
    \item Slice the surface with a plane where $x = c$ (constant)
    \item You get a different 2D curve
    \item $\frac{\partial f}{\partial y}$ is the slope of that curve
\end{itemize}
\end{columns}

\vspace{0.5cm}

\textbf{Let's visualize this in Python!}

\end{frame}

%-----------------------------------------------
\begin{frame}[fragile]{Python: Plotting $f(x,y) = x^2 + y^2$}

\begin{lstlisting}
import numpy as np
import matplotlib.pyplot as plt
from mpl_toolkits.mplot3d import Axes3D

# Create grid of x and y values
x = np.linspace(-3, 3, 50)
y = np.linspace(-3, 3, 50)
X, Y = np.meshgrid(x, y)

# Compute the function
Z = X**2 + Y**2

# Create 3D plot
fig = plt.figure(figsize=(10, 8))
ax = fig.add_subplot(111, projection='3d')
ax.plot_surface(X, Y, Z, cmap='viridis', alpha=0.8)
ax.set_xlabel('x')
ax.set_ylabel('y')
ax.set_zlabel('f(x, y)')
ax.set_title('$f(x, y) = x^2 + y^2$')
plt.show()
\end{lstlisting}

\end{frame}

%-----------------------------------------------
\begin{frame}{The 3D Surface: $f(x,y) = x^2 + y^2$}

\begin{center}
\textit{[Run the Python code to display the paraboloid]}
\end{center}

\vspace{0.5cm}

\textbf{Key observations:}
\begin{itemize}
    \item The surface has a minimum at $(0, 0)$ where $f = 0$
    \item As you move away from the origin in any direction, $f$ increases
    \item The partial derivatives tell us how steep the surface is in each coordinate direction
\end{itemize}

\vspace{0.5cm}

For $f(x,y) = x^2 + y^2$:
\[
\frac{\partial f}{\partial x} = 2x \qquad \frac{\partial f}{\partial y} = 2y
\]

At the point $(1, 2)$: $\frac{\partial f}{\partial x} = 2$ and $\frac{\partial f}{\partial y} = 4$

\end{frame}

%-----------------------------------------------
\begin{frame}[fragile]{Python: Visualizing Partial Derivatives}

\begin{lstlisting}
# Add a point and tangent directions
fig = plt.figure(figsize=(10, 8))
ax = fig.add_subplot(111, projection='3d')
ax.plot_surface(X, Y, Z, cmap='viridis', alpha=0.6)

# Point (1, 2) on the surface
px, py = 1, 2
pz = px**2 + py**2  # f(1,2) = 5

# Plot the point
ax.scatter([px], [py], [pz], color='red', s=100, label='(1, 2, 5)')

# Partial derivative slopes at (1,2): df/dx = 2, df/dy = 4
ax.set_xlabel('x')
ax.set_ylabel('y')
ax.set_zlabel('f(x, y)')
ax.legend()
plt.show()
\end{lstlisting}

\end{frame}

%-----------------------------------------------
\begin{frame}{Activity Part 1: Your Turn! (15 min)}

\textbf{Problem:} Let $h(x, y) = x^3 - 2xy + y^2$

\vspace{0.5cm}

\textbf{Step 1:} Compute the partial derivatives (on paper):
\begin{enumerate}
    \item Find $\frac{\partial h}{\partial x}$
    \item Find $\frac{\partial h}{\partial y}$
\end{enumerate}

\vspace{0.5cm}

\textbf{Step 2:} Evaluate at the point $(2, 1)$:
\begin{enumerate}
    \item What is $\frac{\partial h}{\partial x}$ at $(2, 1)$?
    \item What is $\frac{\partial h}{\partial y}$ at $(2, 1)$?
\end{enumerate}

\vspace{0.5cm}

\textbf{Step 3:} Plot the surface in Python (see next slide for starter code).

\end{frame}

%-----------------------------------------------
\begin{frame}[fragile]{Activity Part 1: Plotting Code}

\textbf{Modify this code to plot $h(x, y) = x^3 - 2xy + y^2$:}

\begin{lstlisting}
import numpy as np
import matplotlib.pyplot as plt
from mpl_toolkits.mplot3d import Axes3D

x = np.linspace(-3, 3, 50)
y = np.linspace(-3, 3, 50)
X, Y = np.meshgrid(x, y)

# TODO: Define Z = h(x, y)
Z = _______________________  # Fill this in!

fig = plt.figure(figsize=(10, 8))
ax = fig.add_subplot(111, projection='3d')
ax.plot_surface(X, Y, Z, cmap='coolwarm', alpha=0.8)
ax.set_xlabel('x')
ax.set_ylabel('y')
ax.set_zlabel('h(x, y)')
ax.set_title('$h(x, y) = x^3 - 2xy + y^2$')
plt.show()
\end{lstlisting}

\end{frame}

%-----------------------------------------------
\begin{frame}{Activity Part 1: Solutions}

For $h(x, y) = x^3 - 2xy + y^2$:

\vspace{0.3cm}

\textbf{Partial derivatives:}
\[
\frac{\partial h}{\partial x} = 3x^2 - 2y \qquad \frac{\partial h}{\partial y} = -2x + 2y
\]

\vspace{0.3cm}

\textbf{At the point $(2, 1)$:}
\begin{align*}
\frac{\partial h}{\partial x}\bigg|_{(2,1)} &= 3(2)^2 - 2(1) = 12 - 2 = 10 \\[0.3cm]
\frac{\partial h}{\partial y}\bigg|_{(2,1)} &= -2(2) + 2(1) = -4 + 2 = -2
\end{align*}

\vspace{0.3cm}

\textbf{Python:} \texttt{Z = X**3 - 2*X*Y + Y**2}

\end{frame}

%-----------------------------------------------
\begin{frame}{Beyond Visualization: Higher Dimensions}

So far we can visualize functions of 2 variables as surfaces in 3D.

\vspace{0.5cm}

\textbf{But what about functions of 4, 5, or 1000 variables?}

\vspace{0.3cm}

\begin{itemize}
    \item We can't visualize $f(x_1, x_2, x_3, x_4, x_5)$ as a surface
    \item But partial derivatives still work exactly the same way!
    \item Each $\frac{\partial f}{\partial x_i}$ tells us how $f$ changes when we vary $x_i$ alone
\end{itemize}

\vspace{0.5cm}

\textbf{The math doesn't care about our visual limitations.}

\end{frame}

%-----------------------------------------------
\begin{frame}{Activity Part 2: High Dimensions (10 min)}

\textbf{Problem:} Let $f(x_1, x_2, x_3, x_4, x_5) = x_1^3 + 2x_2^2 + x_3 x_4 + \ln(x_5) + 5x_1$

\vspace{0.5cm}

\textbf{Find all five partial derivatives:}
\begin{enumerate}
    \item $\frac{\partial f}{\partial x_1} = $
    \item $\frac{\partial f}{\partial x_2} = $
    \item $\frac{\partial f}{\partial x_3} = $
    \item $\frac{\partial f}{\partial x_4} = $
    \item $\frac{\partial f}{\partial x_5} = $
\end{enumerate}

\vspace{0.5cm}

\textbf{Evaluate all partial derivatives at the point $(1, 2, 3, 4, 1)$}

\end{frame}

%-----------------------------------------------
\begin{frame}{Activity Part 2: Solutions}

For $f(x_1, x_2, x_3, x_4, x_5) = x_1^3 + 2x_2^2 + x_3 x_4 + \ln(x_5) + 5x_1$

\vspace{0.3cm}

\textbf{Partial derivatives:}
\begin{align*}
\frac{\partial f}{\partial x_1} &= 3x_1^2 + 5 & \frac{\partial f}{\partial x_2} &= 4x_2 \\[0.2cm]
\frac{\partial f}{\partial x_3} &= x_4 & \frac{\partial f}{\partial x_4} &= x_3 \\[0.2cm]
\frac{\partial f}{\partial x_5} &= \frac{1}{x_5}
\end{align*}

\vspace{0.3cm}

\textbf{At $(1, 2, 3, 4, 1)$:}
\begin{align*}
\frac{\partial f}{\partial x_1} &= 3(1)^2 + 5 = 8 & \frac{\partial f}{\partial x_2} &= 4(2) = 8 \\[0.2cm]
\frac{\partial f}{\partial x_3} &= 4 & \frac{\partial f}{\partial x_4} &= 3 \\[0.2cm]
\frac{\partial f}{\partial x_5} &= \frac{1}{1} = 1
\end{align*}

\end{frame}

%-----------------------------------------------
\begin{frame}{The Gradient: Collecting All Partial Derivatives}

We can package all partial derivatives into a single vector called the \textbf{gradient}:

\vspace{0.5cm}

\textbf{Notation:} For $f(x_1, x_2, \ldots, x_n)$, the gradient is:
\[
\nabla f = \begin{pmatrix} \frac{\partial f}{\partial x_1} \\[0.3cm] \frac{\partial f}{\partial x_2} \\[0.3cm] \vdots \\[0.3cm] \frac{\partial f}{\partial x_n} \end{pmatrix}
\]

\vspace{0.3cm}

\textbf{Example:} For our 5-variable function at $(1, 2, 3, 4, 1)$:
\[
\nabla f \big|_{(1,2,3,4,1)} = \begin{pmatrix} 8 \\ 8 \\ 4 \\ 3 \\ 1 \end{pmatrix}
\]

\end{frame}

%-----------------------------------------------
\begin{frame}{Discussion: What Does the Gradient Mean? (10 min)}

\textbf{Think-Pair-Share:}

\vspace{0.5cm}

Consider our 2-variable example: $f(x,y) = x^2 + y^2$ at point $(1, 2)$

The gradient is: $\nabla f = \begin{pmatrix} 2x \\ 2y \end{pmatrix} = \begin{pmatrix} 2 \\ 4 \end{pmatrix}$ at $(1, 2)$

\vspace{0.5cm}

\textbf{Questions to discuss:}
\begin{enumerate}
    \item If you're standing on the surface at $(1, 2, 5)$ and want to walk in a direction that makes $f$ increase as fast as possible, which direction should you go?
    
    \item What direction would make $f$ \textit{decrease} as fast as possible?
    
    \item What if you walked perpendicular to the gradient -- what would happen to $f$?
\end{enumerate}

\end{frame}

%-----------------------------------------------
\begin{frame}{Key Insight: The Gradient Points ``Uphill''}

\textbf{Fundamental property of the gradient:}

\vspace{0.3cm}

\begin{center}
\fbox{\parbox{0.85\textwidth}{
The gradient $\nabla f$ at a point is a vector that:
\begin{itemize}
    \item Points in the direction of \textbf{steepest increase} of $f$
    \item Has magnitude equal to the \textbf{rate of steepest increase}
\end{itemize}
}}
\end{center}

\vspace{0.5cm}

\textbf{Consequences:}
\begin{itemize}
    \item $-\nabla f$ points in the direction of steepest \textit{decrease}
    \item Directions perpendicular to $\nabla f$ are ``level'' (no change in $f$)
    \item At a minimum or maximum, $\nabla f = \mathbf{0}$
\end{itemize}

\vspace{0.5cm}

\textbf{This is true regardless of how many variables we have!}

\end{frame}

%-----------------------------------------------
\begin{frame}{Summary}

\textbf{Today we learned:}

\vspace{0.3cm}

\begin{enumerate}
    \item \textbf{Partial derivatives} measure how a function changes when we vary one variable while holding all others constant
    
    \vspace{0.2cm}
    
    \item The same differentiation rules (power, log, trig) apply -- just treat other variables as constants
    
    \vspace{0.2cm}
    
    \item For 2-variable functions, we can visualize partial derivatives as slopes of slices through a surface
    
    \vspace{0.2cm}
    
    \item Partial derivatives work identically in any number of dimensions, even when we can't visualize
    
    \vspace{0.2cm}
    
    \item The \textbf{gradient} $\nabla f$ collects all partial derivatives into a vector that points in the direction of steepest increase
\end{enumerate}

\end{frame}

%-----------------------------------------------
\begin{frame}{Looking Ahead}

\textbf{Why does this matter?}

\vspace{0.5cm}

The gradient will become essential as we move forward. For now, remember:

\vspace{0.3cm}

\begin{itemize}
    \item Functions of many variables are everywhere in computation
    \item The gradient tells us which way is ``uphill'' on any surface
    \item This works in 2 dimensions, 5 dimensions, or 5 million dimensions
\end{itemize}

\vspace{0.5cm}

\textbf{Next time:} We'll see how the gradient connects to optimization.

\end{frame}

\end{document}
